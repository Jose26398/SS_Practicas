\documentclass[11pt,a4paper]{report}
\usepackage[spanish,es-nodecimaldot]{babel}	% Utilizar español
\usepackage[utf8]{inputenc}					% Caracteres UTF-8
\usepackage{graphicx}						% Imagenes
\usepackage[hidelinks]{hyperref}			% Poner enlaces sin marcarlos en rojo
\usepackage{fancyhdr}						% Modificar encabezados y pies de pagina
\usepackage{float}							% Insertar figuras
\usepackage[textwidth=390pt]{geometry}		% Anchura de la pagina
\usepackage[nottoc]{tocbibind}				% Referencias (no incluir num pagina indice en Indice)
\usepackage{enumitem}						% Permitir enumerate con distintos simbolos
\usepackage[T1]{fontenc}					% Usar textsc en sections
\usepackage{amsmath}						% Símbolos matemáticos

% Comando para poner el nombre de la asignatura
\newcommand{\asignatura}{Simulación de Sistemas}
\newcommand{\autor}{José María Sánchez Guerrero}
\newcommand{\titulo}{Práctica 2}
\newcommand{\subtitulo}{Modelos de Monte Carlo. Generadores de datos}

% Configuracion de encabezados y pies de pagina
\pagestyle{fancy}
\lhead{\autor{}}
\rhead{\asignatura{}}
\lfoot{Grado en Ingeniería Informática}
\cfoot{}
\rfoot{\thepage}
\renewcommand{\headrulewidth}{0.4pt}		% Linea cabeza de pagina
\renewcommand{\footrulewidth}{0.4pt}		% Linea pie de pagina

\begin{document}
\pagenumbering{gobble}

% Pagina de titulo
\begin{titlepage}

\begin{minipage}{\textwidth}

\centering

\includegraphics[scale=0.5]{img/ugr.png}\\

\textsc{\Large \asignatura{}\\[0.2cm]}
\textsc{GRADO EN INGENIERÍA INFORMÁTICA}\\[1cm]

\noindent\rule[-1ex]{\textwidth}{1pt}\\[1.5ex]
\textsc{{\Huge \titulo\\[0.5ex]}}
\textsc{{\Large \subtitulo\\}}
\noindent\rule[-1ex]{\textwidth}{2pt}\\[3.5ex]

\end{minipage}

\vspace{0.5cm}

\begin{minipage}{\textwidth}

\centering

\textbf{Autor}\\ {\autor{}}\\[2.5ex]
\textbf{Rama}\\ {Computación y Sistemas Inteligentes}\\[2.5ex]
\vspace{0.3cm}

\includegraphics[scale=0.3]{img/etsiit.jpeg}

\vspace{0.7cm}
\textsc{Escuela Técnica Superior de Ingenierías Informática y de Telecomunicación}\\
\vspace{1cm}
\textsc{Curso 2019-2020}
\end{minipage}
\end{titlepage}

\pagenumbering{arabic}
\tableofcontents
\thispagestyle{empty}				% No usar estilo en la pagina de indice

\newpage

\setlength{\parskip}{1em}

\chapter{Mi segundo modelo de simulación de MonteCarlo}

Un establecimiento se abastece diariamente de un cierto producto, y necesita decidir cuántas unidades \textbf{s} de ese producto
pedir cada día Se desea encontrar el valor de $s$ donde se maximice la ganancia esperada. Obtenemos una ganancia de \textbf{x}
euros por unidad vendida, y una pérdida de \textbf{y} euros si no se ha vendido al final del dia. También contaremos con un valor
de demanda, el cual serán los productos solicitados cada día y seguirán una distribución de probabilidad \textbf{P} con el cual
jugaremos para ver los distintos resultados que nos ofrecerán.


\section{Experimentación inicial}

Empezaremos por un modelo de MonteCarlo inicial, en el cual representaremos todas las variables anteriores en el código para calcular
la ganancia, y también dispondremos de varias distribuciones para calcular la demanda de productos solicitados. La ganancia viene
determinada por lo siguiente:

\begin{equation}
	g(s,x,y,d)=\left\{\begin{matrix}
	x*s\hspace{27mm}si\hspace{2mm}d\geq s\\ 
	x*d-(s-d)*y\hspace{6mm}si\hspace{2mm}d\geq s
	\end{matrix}\right.
\end{equation}

Y tendremos estas tres tipos de distribuciones que podremos seleccionar en el código la que queramos. Hay que tener en cuenta que la
simulación será para 100 valores de $s$, por lo que esto influirá también en las distribuciones:

\begin{itemize}
	\item $P(D=d)$ se distribuye uniformemente entre 0 y 99
	\item $P(D=d)$ es proporcional a $100-d$, $\forall d=0,1,2,...,99$
	\item $P(D=d)$ tiene una distribución "triangular" que crece con $d/2500$ entre 0 y 50; y decrece con $(100-d)/2500$ entre 50 y 99.
\end{itemize}

Por último, como vamos a ejecutar un modelo varias \textbf{veces}, tendremos que obtener la media para cada $s$ y también la desviación
típica con la fórmula
\begin{equation}
	desviaciont=\sqrt{\frac{sum2-veces*gananciaesperada*gananciaesperada}{veces-1}}
\end{equation}

El código de este ejercicio está en el archivo $generadores.c$, proporcionado junto a la memoria. Vamos a ejecutarlo con distintos parámetros
para ver cómo afectan al modelo. La ejecución la vamos a dividir en tres: una para $x=10, y=1$; otra para $x=10, y=5$; y la última para
$x=10, y=10$. Los valores de las tablas serán tanto la ganancia como la desviación típica para los distintos valores de $veces$ y
contrastaremos los resultados.

Estos han sido los resultados para la distribución \textbf{uniforme}:

\begin{table}[H]
	\parbox{.45\linewidth}{
	\centering
	\begin{tabular}{c|ccc}
	\textbf{Veces} & \textbf{s} & \textbf{Ganancia} & \textbf{Desviación} \\ \hline
	\textbf{100}   & 86         & 495.68            & 316.09			  \\ \hline
	\textbf{1000}  & 91         & 472.38            & 308.29			  \\ \hline
	\textbf{5000}  & 90         & 457.66            & 310.85 			  \\ \hline
	\textbf{10000} & 91         & 454.09            & 311.94			  \\ \hline
	\textbf{100000}& 87         & 450.85            & 304.69			  \\
	\end{tabular}
	\caption{$x=10$, $y=1$}
	}
	\hfill
	\parbox{.45\linewidth}{
	\centering
	\begin{tabular}{c|ccc}
	\textbf{Veces} & \textbf{s} & \textbf{Ganancia} & \textbf{Desviación} \\ \hline
	\textbf{100}   & 77         & 390.80            & 388.73			  \\ \hline
	\textbf{1000}  & 70         & 347.14            & 348.71			  \\ \hline
	\textbf{5000}  & 67         & 335.88            & 332.22 			  \\ \hline
	\textbf{10000} & 65         & 333.65            & 323.97			  \\ \hline
	\textbf{100000}& 66         & 330.05            & 331.99			  \\
	\end{tabular}
	\caption{$x=10$, $y=5$}
	}
\end{table}
\begin{table}[H]
	\centering
	\begin{tabular}{c|ccc}
	\textbf{Veces} & \textbf{s} & \textbf{Ganancia} & \textbf{Desviación} \\ \hline
	\textbf{100}   & 50         & 290.60            & 314.91			  \\ \hline
	\textbf{1000}  & 55         & 268.84            & 356.97			  \\ \hline
	\textbf{5000}  & 48         & 254.50            & 306.41 			  \\ \hline
	\textbf{10000} & 53         & 249.94            & 345.77			  \\ \hline
	\textbf{100000}& 51         & 247.03            & 333.85			  \\
	\end{tabular}
	\caption{$x=10$, $y=10$}
\end{table}

Como podemos ver..............................



Estos han sido los resultados para la distribución \textbf{proporcional}:

\begin{table}[H]
	\parbox{.45\linewidth}{
	\centering
	\begin{tabular}{c|ccc}
	\textbf{Veces} & \textbf{s} & \textbf{Ganancia} & \textbf{Desviación} \\ \hline
	\textbf{100}   & 78         & 322.40            & 285.75			  \\ \hline
	\textbf{1000}  & 69         & 297.04            & 243.19			  \\ \hline
	\textbf{5000}  & 71         & 289.95            & 244.48 			  \\ \hline
	\textbf{10000} & 72         & 287.13            & 227.74			  \\ \hline
	\textbf{100000}& 67         & 283.75            & 235.86			  \\
	\end{tabular}
	\caption{$x=10$, $y=1$}
	}
	\hfill
	\parbox{.45\linewidth}{
	\centering
	\begin{tabular}{c|ccc}
	\textbf{Veces} & \textbf{s} & \textbf{Ganancia} & \textbf{Desviación} \\ \hline
	\textbf{100}   & 57         & 233.40            & 293.49			  \\ \hline
	\textbf{1000}  & 44         & 202.47            & 229.51			  \\ \hline
	\textbf{5000}  & 39         & 193.12            & 203.07 			  \\ \hline
	\textbf{10000} & 42         & 193.25            & 221.17			  \\ \hline
	\textbf{100000}& 44         & 188.60            & 231.17			  \\
	\end{tabular}
	\caption{$x=10$, $y=5$}
	}
\end{table}
\begin{table}[H]
	\centering
	\begin{tabular}{c|ccc}
	\textbf{Veces} & \textbf{s} & \textbf{Ganancia} & \textbf{Desviación} \\ \hline
	\textbf{100}   & 30         & 167.40            & 210.69			  \\ \hline
	\textbf{1000}  & 30         & 146.60            & 201.91			  \\ \hline
	\textbf{5000}  & 27         & 136.17            & 179.10 			  \\ \hline
	\textbf{10000} & 27         & 135.01            & 178.65			  \\ \hline
	\textbf{100000}& 28         & 133.66            & 189.31			  \\
	\end{tabular}
	\caption{$x=10$, $y=10$}
\end{table}

Como podemos ver.......

Estos han sido los resultados para la distribución \textbf{triangular}:

\begin{table}[H]
	\parbox{.45\linewidth}{
	\centering
	\begin{tabular}{c|ccc}
	\textbf{Veces} & \textbf{s} & \textbf{Ganancia} & \textbf{Desviación} \\ \hline
	\textbf{100}   & 83         & 493.40            & 217.66			  \\ \hline
	\textbf{1000}  & 77         & 476.39            & 212.23			  \\ \hline
	\textbf{5000}  & 76         & 466.11            & 205.99 			  \\ \hline
	\textbf{10000} & 76         & 466.78            & 206.22			  \\ \hline
	\textbf{100000}& 79         & 464.94            & 212.76			  \\
	\end{tabular}
	\caption{$x=10$, $y=1$}
	}
	\hfill
	\parbox{.45\linewidth}{
	\centering
	\begin{tabular}{c|ccc}
	\textbf{Veces} & \textbf{s} & \textbf{Ganancia} & \textbf{Desviación} \\ \hline
	\textbf{100}   & 74         & 432.50            & 282.03			  \\ \hline
	\textbf{1000}  & 62         & 408.09            & 225.36			  \\ \hline
	\textbf{5000}  & 59         & 389.14            & 218.78 			  \\ \hline
	\textbf{10000} & 58         & 388.80            & 215.32			  \\ \hline
	\textbf{100000}& 59         & 386.79            & 221.56 			  \\
	\end{tabular}
	\caption{$x=10$, $y=5$}
	}
\end{table}
\begin{table}[H]
	\centering
	\begin{tabular}{c|ccc}
	\textbf{Veces} & \textbf{s} & \textbf{Ganancia} & \textbf{Desviación} \\ \hline
	\textbf{100}   & 51         & 367.00            & 225.96			  \\ \hline
	\textbf{1000}  & 44         & 335.26            & 187.24			  \\ \hline
	\textbf{5000}  & 53         & 336.00            & 251.46 			  \\ \hline
	\textbf{10000} & 48         & 337.11            & 217.58			  \\ \hline
	\textbf{100000}& 49         & 333.64            & 228.09			  \\
	\end{tabular}
	\caption{$x=10$, $y=10$}
\end{table}

Vemos que-..................






\section{Modificaciones del modelo}

\subsection{Cantidad fija de devolución}

En este apartado vamos a modificar el modelo construido anteriormente, de tal forma que el establecimiento pueda devolver las unidades
no vendidas. De esta forma hay que pagar una cantidad fija de $z$ euros de gastos de devolución de las unidades no vendidas, en vez de
tener una pérdida de $y$ euros por unidad. Esta cantidad no varía, a menos que la $z=0$.

La función de ganancia ahora sería:
\begin{equation}
	g(s,x,y,d)=\left\{\begin{matrix}
	x*s\hspace{12mm}si\hspace{2mm}d\geq s\\ 
	x*d-z\hspace{5mm}si\hspace{2mm}d\geq s
	\end{matrix}\right.
\end{equation}

Estará implementada en el código $generadoresModificados.c$ y las pruebas que vamos a realizar con él van a ser las mismas que hicimos
anteriormente, y asi poder comparar unos resultados con otros. Las ejecuciones nos han dado las siguientes tablas:

\begin{table}[H]
	\centering
	\begin{tabular}{c|cccc}
	\textbf{\hspace{5mm}z\hspace{5mm}}   & \textbf{Distribución} & \textbf{\hspace{5mm}s\hspace{5mm}} & \textbf{Ganancia} & \textbf{Desviación} \\ \hline
	\textbf{5}   & uniforme 			 & 95         & 490.45            & 297.18 			  \\
	\textbf{5}   & proporcional			 & 91         & 326.41            & 236.09			  \\
	\textbf{5}   & triangular 			 & 95         & 495.37            & 203.63			  \\ \hline
	\textbf{400} & uniforme 			 & 59         & 178.91            & 367.37 			  \\
	\textbf{400} & proporcional			 & 20         & 18.58             & 246.17			  \\
	\textbf{400} & triangular 		     & 44         & 233.82            & 273.39			  \\ \hline
	\textbf{200} & uniforme 			 & 79         & 318.07            & 317.13 			  \\
	\textbf{200} & proporcional		     & 61         & 142.47            & 255.30			  \\
	\textbf{200} & triangular 		     & 55         & 323.75            & 214.21			  \\
	\end{tabular}
	\caption{Resultados para todas las distribuciones con $x=10$ y $veces=100000$}
\end{table}



\subsection{Cantidad relativa de devolución}

Por último, vamos a 'fusionar' los dos últimos casos en uno. Si el valor $z$ es relativamente grande, no interesará pagar esa cantidad
de dinero cuando queden pocas unidades sin vender. Por otro lado, cuando el número de unidades no vendidas sea pequeño, es preferible asumir
la pérdidas de $y$ que tener que pagar los gastos de devolución.

La función de la ganancia se nos quedaría de la siguiente forma:
\begin{equation}
	g(s,x,y,d)=\left\{\begin{matrix}
	x*s\hspace{37mm}si\hspace{2mm}d\geq s\\ 
	x*d-min\{z,(s-d)*y\}\hspace{4mm}si\hspace{2mm}d\geq s
	\end{matrix}\right.
\end{equation}

También estará implementada en $generadoresModificados.c$, y para ejecutarla, tendremos que cambiar el parámetro $modificacion=2$. El resto
de valores, se mantendrán exactamente iguales que en las ejjecuciones anteriores. Los resultados han sido los siguientes:

\begin{table}[H]
	\centering
	\begin{tabular}{c|cccc}
	\textbf{\hspace{5mm}z\hspace{5mm}}   & \textbf{Distribución} & \textbf{\hspace{5mm}s\hspace{5mm}} & \textbf{Ganancia} & \textbf{Desviación} \\ \hline
	\textbf{100} & uniforme 			 & 92         & 411.21            & 309.86 			  \\
	\textbf{100} & proporcional			 & 72         & 238.44            & 247.52			  \\
	\textbf{100} & triangular 			 & 74         & 415.98            & 217.21			  \\ \hline
	\textbf{150} & uniforme 			 & 86         & 380.06            & 327.23 			  \\
	\textbf{150} & proporcional			 & 56         & 206.16            & 249.46			  \\
	\textbf{150} & triangular 		     & 61         & 396.93            & 211.83			  \\ \hline
	\textbf{200} & uniforme 			 & 81         & 356.57            & 317.13 			  \\
	\textbf{200} & proporcional		     & 43         & 189.64            & 225.30			  \\
	\textbf{200} & triangular 		     & 60         & 390.21            & 219.63			  \\
	\end{tabular}
	\caption{Resultados para todas las distribuciones con $x=10$, $y=5$ y $veces=100000$}
\end{table}

Como podemos ver....

\begin{table}[H]
	\centering
	\begin{tabular}{c|cccc}
	\textbf{\hspace{5mm}y\hspace{5mm}}   & \textbf{Distribución} & \textbf{\hspace{5mm}s\hspace{5mm}} & \textbf{Ganancia} & \textbf{Desviación} \\ \hline
	\textbf{3} & uniforme 			 	 & 79         & 381.69            & 334.20 			  \\
	\textbf{3} & proporcional			 & 54         & 225.98            & 240.16			  \\
	\textbf{3} & triangular 			 & 64         & 419.09            & 210.75			  \\ \hline
	\textbf{7} & uniforme 			 	 & 83         & 344.55            & 337.87 			  \\
	\textbf{7} & proporcional		     & 51         & 172.44            & 258.79			  \\
	\textbf{7} & triangular 		     & 59         & 373.18            & 225.49			  \\
	\textbf{10} & uniforme 				 & 81         & 336.49            & 335.62 			  \\
	\textbf{10} & proporcional			 & 53         & 162.06            & 260.66			  \\
	\textbf{10} & triangular 		     & 57         & 358.21            & 222.97			  \\ \hline
	\end{tabular}
	\caption{Resultados para todas las distribuciones con $x=10$, $z=200$ y $veces=100000$}
\end{table}

Podemos ver que las tablas...........



\chapter{Generadores de datos}

Este modelo de simulación lo vamos a tratar con el problema del \textit{aparcamiento}, en el cual un coche se dispone
a aparcar a una distancia $x$ de su destino. También dispondremos de variables como el número de plazas que alcanza a
ver el conductor desde su posición o la probabilidad de que esa plaza esté ocupada o no. El ejercicio consiste en elegir
una plaza de aparcamiento $c$ en la cual el conductor, ni se qu


\section{Mejorando los generadores}

Para hacernos una idea de los valores que obtendremos y los parámetros que más afectan al rendimiento de este modelo,
vamos a realizar una ejecución inici

\subsection{Generadores ordenados}

\subsection{Implementación con búsqueda binaria}

\subsection{Mejora de eficiencia en el generador a}


\section{Generadores congruenciales}

Para hacernos una idea de los valores que obtendremos y los parámetros que más afectan al rendimiento de este modelo,
vamos a realizar una ejecución inici


\end{document}

