\documentclass[11pt,a4paper]{report}
\usepackage[spanish,es-nodecimaldot]{babel}	% Utilizar español
\usepackage[utf8]{inputenc}					% Caracteres UTF-8
\usepackage{graphicx}						% Imagenes
\usepackage[hidelinks]{hyperref}			% Poner enlaces sin marcarlos en rojo
\usepackage{fancyhdr}						% Modificar encabezados y pies de pagina
\usepackage{float}							% Insertar figuras
\usepackage[textwidth=390pt]{geometry}		% Anchura de la pagina
\usepackage[nottoc]{tocbibind}				% Referencias (no incluir num pagina indice en Indice)
\usepackage{enumitem}						% Permitir enumerate con distintos simbolos
\usepackage[T1]{fontenc}					% Usar textsc en sections
\usepackage{amsmath}						% Símbolos matemáticos

% Comando para poner el nombre de la asignatura
\newcommand{\asignatura}{Simulación de Sistemas}
\newcommand{\autor}{José María Sánchez Guerrero}
\newcommand{\titulo}{Práctica 2}
\newcommand{\subtitulo}{Modelos de Monte Carlo. Generadores de datos}

% Configuracion de encabezados y pies de pagina
\pagestyle{fancy}
\lhead{\autor{}}
\rhead{\asignatura{}}
\lfoot{Grado en Ingeniería Informática}
\cfoot{}
\rfoot{\thepage}
\renewcommand{\headrulewidth}{0.4pt}		% Linea cabeza de pagina
\renewcommand{\footrulewidth}{0.4pt}		% Linea pie de pagina

\begin{document}
\pagenumbering{gobble}

% Pagina de titulo
\begin{titlepage}

\begin{minipage}{\textwidth}

\centering

\includegraphics[scale=0.5]{img/ugr.png}\\

\textsc{\Large \asignatura{}\\[0.2cm]}
\textsc{GRADO EN INGENIERÍA INFORMÁTICA}\\[1cm]

\noindent\rule[-1ex]{\textwidth}{1pt}\\[1.5ex]
\textsc{{\Huge \titulo\\[0.5ex]}}
\textsc{{\Large \subtitulo\\}}
\noindent\rule[-1ex]{\textwidth}{2pt}\\[3.5ex]

\end{minipage}

\vspace{0.5cm}

\begin{minipage}{\textwidth}

\centering

\textbf{Autor}\\ {\autor{}}\\[2.5ex]
\textbf{Rama}\\ {Computación y Sistemas Inteligentes}\\[2.5ex]
\vspace{0.3cm}

\includegraphics[scale=0.3]{img/etsiit.jpeg}

\vspace{0.7cm}
\textsc{Escuela Técnica Superior de Ingenierías Informática y de Telecomunicación}\\
\vspace{1cm}
\textsc{Curso 2019-2020}
\end{minipage}
\end{titlepage}

\pagenumbering{arabic}
\tableofcontents
\thispagestyle{empty}				% No usar estilo en la pagina de indice

\newpage

\setlength{\parskip}{1em}

\chapter{Mi segundo modelo de simulación de MonteCarlo}

Este modelo de simulación lo vamos a tratar con el problema del \textit{aparcamiento}, en el cual un coche se dispone
a aparcar a una distancia $x$ de su destino. También dispondremos de variables como el número de plazas que alcanza a
ver el conductor desde su posición o la probabilidad de que esa plaza esté ocupada o no. El ejercicio consiste en elegir
una plaza de aparcamiento $c$ en la cual el conductor, ni se quede muy corto ni se pase, es decir, que encuentre un
valor que minimice la distancia esperada desde el lugar de aparcamiento hasta el objetivo.


\section{Experimentación inicial}

Para hacernos una idea de los valores que obtendremos y los parámetros que más afectan al rendimiento de este modelo,
vamos a realizar una ejecución inici


\section{Modificaciones del modelo}

Para hacernos una idea de los valores que obtendremos y los parámetros que más afectan al rendimiento de este modelo,
vamos a realizar una ejecución inici




\chapter{Generadores de datos}

Este modelo de simulación lo vamos a tratar con el problema del \textit{aparcamiento}, en el cual un coche se dispone
a aparcar a una distancia $x$ de su destino. También dispondremos de variables como el número de plazas que alcanza a
ver el conductor desde su posición o la probabilidad de que esa plaza esté ocupada o no. El ejercicio consiste en elegir
una plaza de aparcamiento $c$ en la cual el conductor, ni se qu


\section{Mejorando los generadores}

Para hacernos una idea de los valores que obtendremos y los parámetros que más afectan al rendimiento de este modelo,
vamos a realizar una ejecución inici

\subsection{Generadores ordenados}

\subsection{Implementación con búsqueda binaria}

\subsection{Mejora de eficiencia en el generador a}


\section{Generadores congruenciales}

Para hacernos una idea de los valores que obtendremos y los parámetros que más afectan al rendimiento de este modelo,
vamos a realizar una ejecución inici


\end{document}

