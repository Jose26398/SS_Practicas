\documentclass[11pt,a4paper]{report}
\usepackage[spanish,es-nodecimaldot]{babel}	% Utilizar español
\usepackage[utf8]{inputenc}					% Caracteres UTF-8
\usepackage{graphicx}						% Imagenes
\usepackage[hidelinks]{hyperref}			% Poner enlaces sin marcarlos en rojo
\usepackage{fancyhdr}						% Modificar encabezados y pies de pagina
\usepackage{float}							% Insertar figuras
\usepackage[textwidth=390pt]{geometry}		% Anchura de la pagina
\usepackage[nottoc]{tocbibind}				% Referencias (no incluir num pagina indice en Indice)
\usepackage{enumitem}						% Permitir enumerate con distintos simbolos
\usepackage[T1]{fontenc}					% Usar textsc en sections
\usepackage{amsmath}						% Símbolos matemáticos
\usepackage{listings}
\usepackage{color}

 
\definecolor{codegreen}{rgb}{0,0.6,0}
\definecolor{codegray}{rgb}{0.5,0.5,0.5}
\definecolor{codepurple}{rgb}{0.58,0,0.82}
\definecolor{backcolour}{rgb}{0.95,0.95,0.95}
 
\lstdefinestyle{mystyle}{
    backgroundcolor=\color{backcolour},   
    commentstyle=\color{codegreen},
    keywordstyle=\color{magenta},
    numberstyle=\tiny\color{codegray},
    stringstyle=\color{codepurple},
    basicstyle=\footnotesize,
    breakatwhitespace=false,         
    breaklines=true,                 
    captionpos=b,                    
    keepspaces=true,                 
    numbers=left,                    
    numbersep=5pt,                  
    showspaces=false,                
    showstringspaces=false,
    showtabs=false,                  
    tabsize=2
}
 
\lstset{style=mystyle, language=C++}

% Comando para poner el nombre de la asignatura
\newcommand{\asignatura}{Simulación de Sistemas}
\newcommand{\autor}{José María Sánchez Guerrero}
\newcommand{\titulo}{Práctica 3}
\newcommand{\subtitulo}{Modelos de Simulación Dinámicos y Discretos}

% Configuracion de encabezados y pies de pagina
\pagestyle{fancy}
\lhead{\autor{}}
\rhead{\asignatura{}}
\lfoot{Grado en Ingeniería Informática}
\cfoot{}
\rfoot{\thepage}
\renewcommand{\headrulewidth}{0.4pt}		% Linea cabeza de pagina
\renewcommand{\footrulewidth}{0.4pt}		% Linea pie de pagina

\begin{document}
\pagenumbering{gobble}

% Pagina de titulo
\begin{titlepage}

\begin{minipage}{\textwidth}

\centering

\includegraphics[scale=0.5]{img/ugr.png}\\

\textsc{\Large \asignatura{}\\[0.2cm]}
\textsc{GRADO EN INGENIERÍA INFORMÁTICA}\\[1cm]

\noindent\rule[-1ex]{\textwidth}{1pt}\\[1.5ex]
\textsc{{\Huge \titulo\\[0.5ex]}}
\textsc{{\Large \subtitulo\\}}
\noindent\rule[-1ex]{\textwidth}{2pt}\\[3.5ex]

\end{minipage}

\vspace{0.5cm}

\begin{minipage}{\textwidth}

\centering

\textbf{Autor}\\ {\autor{}}\\[2.5ex]
\textbf{Rama}\\ {Computación y Sistemas Inteligentes}\\[2.5ex]
\vspace{0.3cm}

\includegraphics[scale=0.3]{img/etsiit.jpeg}

\vspace{0.7cm}
\textsc{Escuela Técnica Superior de Ingenierías Informática y de Telecomunicación}\\
\vspace{1cm}
\textsc{Curso 2019-2020}
\end{minipage}
\end{titlepage}

\pagenumbering{arabic}
\tableofcontents
\thispagestyle{empty}				% No usar estilo en la pagina de indice

\newpage

\setlength{\parskip}{1em}

\chapter{Mi segundo modelo de simulación Discreto}

Nuestro modelo de simulación consistirá en un servidor que presta un determinado servicio a una serie de clientes, los cuales solicitarán dicho servicio
periódicamente. Cuando llega un cliente y el servidor no está ocupado, será atendido inmediatamente; en caso contrario, el cliente tendrá que esperar en
la cola. Cuando se completa un servicio, el servidor elegirá al siguiente en una forma FIFO.

Al empezar la simulación, no habrá clientes esperando y el servidor está libre. Utilizaremos el mismo generador exponencial tanto para el tiempo que tardarán
en llegar los clientes, como el tiempo que tardará el servidor en atender a cada uno.


\section{Simulación con incremento fijo de tiempo}

En esta simulación, vamos a tratar al tiempo incrementándolo de unidad en unidad. Para evitar problemas con el manejo del tiempo, tendremos que modificar
los generadores de datos para que nos devuelvan los valores redondeados al entero más próximo. Si obtenemos un valor igual a 0, devolveremos 1 en su lugar,
ya que el suceso generado quedaría en un tiempo anterior al actual, que generamos al incrementar en una unidad.

Este será nuestro código resultante, que nos servirá tanto para generar el tiempo de llegada del cliente como para generar el timepo del servicio (sólo
tendremos que modificar la variable $tlleg$ por $tserv$):
\newpage
\begin{lstlisting}
float generallegada(float tlleg){
	float u = random();         // o tambien rand() en lugar de random()
	u = ( u / (RAND_MAX+1.0) ); //RAND_MAX es una constante del sistema
	u = round( -tlleg * log(1-u) );

	if (u != 0)
		return u;
	else
		return 1.0;
}
\end{lstlisting}


Para la simulación vamos a emplear diferentes unidades de medida de tiempo (horas, minutos, segundos...) y con un número de clientes a atender bastante alto,
de unos 10.000, para que los resultados sean robustos. Este es el resultado que obtenemos:

Tabla con los resultados....

Como podemos ver...........


\section{Simulación con incremento variable de tiempo}


\end{document}
