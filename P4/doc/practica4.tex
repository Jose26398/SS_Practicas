\documentclass[11pt,a4paper]{article}
\usepackage[spanish,es-nodecimaldot]{babel}	% Utilizar español
\usepackage[utf8]{inputenc}					% Caracteres UTF-8
\usepackage{graphicx}						% Imagenes
\usepackage[hidelinks]{hyperref}			% Poner enlaces sin marcarlos en rojo
\usepackage{fancyhdr}						% Modificar encabezados y pies de pagina
\usepackage{float}							% Insertar figuras
\usepackage[textwidth=390pt]{geometry}		% Anchura de la pagina
\usepackage[nottoc]{tocbibind}				% Referencias (no incluir num pagina indice en Indice)
\usepackage{enumitem}						% Permitir enumerate con distintos simbolos
\usepackage[T1]{fontenc}					% Usar textsc en sections
\usepackage{amsmath}				% Símbolos matemáticos
\usepackage{listings}
\usepackage{color}

 
\definecolor{codegreen}{rgb}{0,0.6,0}
\definecolor{codegray}{rgb}{0.5,0.5,0.5}
\definecolor{codepurple}{rgb}{0.58,0,0.82}
\definecolor{backcolour}{rgb}{0.95,0.95,0.95}
 
\lstdefinestyle{mystyle}{
    backgroundcolor=\color{backcolour},   
    commentstyle=\color{codegreen},
    keywordstyle=\color{magenta},
    numberstyle=\tiny\color{codegray},
    stringstyle=\color{codepurple},
    basicstyle=\footnotesize,
    breakatwhitespace=false,         
    breaklines=true,                 
    captionpos=b,                    
    keepspaces=true,                 
    numbers=left,                    
    numbersep=5pt,                  
    showspaces=false,                
    showstringspaces=false,
    showtabs=false,                  
    tabsize=2
}
 
\lstset{style=mystyle, language=C++}

% Comando para poner el nombre de la asignatura
\newcommand{\asignatura}{Simulación de Sistemas}
\newcommand{\autor}{José María Sánchez Guerrero}
\newcommand{\titulo}{Práctica 4}
\newcommand{\subtitulo}{Modelos de Simulación Dinámicos Continuos}

% Configuracion de encabezados y pies de pagina
\pagestyle{fancy}
\lhead{\autor{}}
\rhead{\asignatura{}}
\lfoot{Grado en Ingeniería Informática}
\cfoot{}
\rfoot{\thepage}
\renewcommand{\headrulewidth}{0.4pt}		% Linea cabeza de pagina
\renewcommand{\footrulewidth}{0.4pt}		% Linea pie de pagina

\begin{document}
\pagenumbering{gobble}

% Pagina de titulo
\begin{titlepage}

\begin{minipage}{\textwidth}

\centering

\includegraphics[scale=0.5]{img/ugr.png}\\

\textsc{\Large \asignatura{}\\[0.2cm]}
\textsc{GRADO EN INGENIERÍA INFORMÁTICA}\\[1cm]

\noindent\rule[-1ex]{\textwidth}{1pt}\\[1.5ex]
\textsc{{\Huge \titulo\\[0.5ex]}}
\textsc{{\Large \subtitulo\\}}
\noindent\rule[-1ex]{\textwidth}{2pt}\\[3.5ex]

\end{minipage}

\vspace{0.5cm}

\begin{minipage}{\textwidth}

\centering

\textbf{Autor}\\ {\autor{}}\\[2.5ex]
\textbf{Rama}\\ {Computación y Sistemas Inteligentes}\\[2.5ex]
\vspace{0.3cm}

\includegraphics[scale=0.3]{img/etsiit.jpeg}

\vspace{0.7cm}
\textsc{Escuela Técnica Superior de Ingenierías Informática y de Telecomunicación}\\
\vspace{1cm}
\textsc{Curso 2019-2020}
\end{minipage}
\end{titlepage}

\pagenumbering{arabic}
\tableofcontents
\thispagestyle{empty}				% No usar estilo en la pagina de indice

\newpage

\setlength{\parskip}{1em}

\section{Introducción}

En esta práctica nos enfrentamos al modelo de ecosistema de Lotka-Volterra, el cual pretende estudiar el crecimiento entre dos especies
relacionadas entre sí. De estas dos, una de ellas serán los \textbf{depredadores ($x$)} y la otra las \textbf{presas ($y$)}.

Hay varias cosas a tener en cuenta en el modelo. Por ejemplo, en ausencia de presas, la población de depredadores disminuye; mientras que
en ausencia de depredadores, la población de presas aumenta. Tampoco tendremos factores de autoinhibición. Las ecuaciones de crecimiento
de ambas poblaciones se pueden describir de la siguiente forma:
$$\frac{\partial x}{\partial t} = a_{11}x-a_{12}xy$$
$$\frac{\partial y}{\partial t} = a_{21}xy-a_{22}y$$

Los valores de $a_{11}$ y $a_{22}$ representan las tasas de crecimiento de cada una de las especies, y por otro lado, $a_12$ y $a_21$ son los
factores de inhibición mutuos teniendo en cuenta a la otra población.


\section{Programar modelo}
El modelo anterior estará implementado en el archivo \textit{depredadores.cpp} adjuntado con la memoria. El programa partirá del pseudocódigo
proporcionado y además se adaptará a las especificaciones comentadas. Dispondrá de dos métodos de derivación diferentes, uno sencillo utilizando
Euler (aunque no sea recomendable), y otro utilizando el método de Runge-Kuta.

Los parámetros que le podremos pasar al programa será los 4 $a_{ij}$ comentados y los valores iniciales para los depredadores y las presas.
Junto con este programa, también voy a adjuntar un script en python que, utilizando este modelo y los parámetros deseados, nos muestra una
gráfica con la evolución de la población entre el instante inicial y el final.

A continuación, vamos a realizar varias pruebas de nuestro modelo para comprobar su funcionamiento.


\newpage
\section{Experimentación inicial}

Podremos estimar las constantes de las ecuaciones con ayuda de las siguientes suposiciones:
\begin{itemize}
	\item Cada pareja de conejos engendra un promedio de 10 crías por año.
	\item Cada zorro captura un promedio de 25 conejos por año.
	\item La edad promedio de los zorros es de 5 años (20\% de los zorros muere anualmente).
	\item En media, el número medio de zorros jóvenes que sobreviven es igual al número de conejos dividido por 25.
\end{itemize}

Con esto tenemos que los valores iniciales para cada una de las constantes son: $a_{11}=5$, $a_{12}=0.05$, $a_{21}=0.0004$ y $a_{22}=0.2$.
Las poblaciones iniciales para los depredadores y las presas serán las siguientes:
$$depredadores(x) = \frac{a_{22}}{a_{21}} = 100$$
$$presas(y) = \frac{a_{11}}{a_{12}} = 500$$

Utilizaremos el método de integración de Runge-Kuta, junto con un intervalo de cálculo de $h=0.1$. El resultado ha sido el siguiente:
\begin{figure}[H]
\centering
\includegraphics[scale=0.65]{img/2-100-500.png}
\caption{depredadores x->100,   presas y->510}
\end{figure}

Como podemos observar.............

Vamos a distanciar progresivamente los valores de $x$ e $y$ para ver cómo evoluciona la población en esos casos. Probaremos aumentando las
presas en 10, disminuyendo los depredadores en 10, haciendo las dos anteriores juntas, y aumentando en 50 las presas y disminuyendo en 50
los depredadores. Las gráficas resultantes han sido las siguientes:
\begin{figure}[H]
	\centering
	\begin{minipage}{0.5\textwidth}
	  \centering
	  \includegraphics[scale=0.4]{img/2-100-510.png}
	  \caption{x=100,  y=510}
	\end{minipage}%
	\begin{minipage}{0.5\textwidth}
	  \centering
	  \includegraphics[scale=0.4]{img/2-90-500.png}
	  \caption{x=90,  y=510}
	\end{minipage}
\end{figure}

\begin{figure}[H]
	\centering
	\begin{minipage}{0.5\textwidth}
	  \centering
	  \includegraphics[scale=0.4]{img/2-90-510.png}
	  \caption{x=90,  y=510}
	\end{minipage}%
	\begin{minipage}{0.5\textwidth}
	  \centering
	  \includegraphics[scale=0.4]{img/2-50-550.png}
	  \caption{x=50,  y=550}
	\end{minipage}
\end{figure}

En este caso, podemos ver que..........

\end{document}



